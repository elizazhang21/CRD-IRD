% !TEX encoding = UTF-8 Unicode
\documentclass[11pt, notitlepage]{article}
\usepackage{geometry}
\geometry{left=2.5cm,right=2.5cm,top=2.5cm,bottom=2.5cm}
\usepackage{amsmath}
\setlength\parindent{0pt}
\usepackage{amsmath,bm}
\usepackage{amssymb}
\usepackage{amsthm}
\usepackage{mathrsfs}
\usepackage{graphicx}
\usepackage{color}
\usepackage{xcolor}
\usepackage{float}
\usepackage{amsfonts}
\usepackage{cite}
\usepackage{graphicx}
\def\\ln{\mathrm{\ln}}
\def\\exp{\mathrm{\exp}}
\def\\max{\mathrm{\max}}
\usepackage{hyperref}
\hypersetup{
    colorlinks=true,
    linkcolor=blue,
    filecolor=magenta,      
    urlcolor= blue,
}
\urlstyle{same}

\def\mathbi#1{\textbf{\em #1}}            %math bold italic letters(use $\mathbi{A}$ etc.)
\def\d{\mathnormal{d}}   

\title{Credit Risk Derivatives Final}
\author{Elizabeth Zhang (kz942)}     
\date{\today}
\bibliographystyle{plain}

\begin{document}
\maketitle
{\centering
  \textbf {Outline}\par
}
\begin{table}[H]
\centering
\scalebox{1.2}{
\begin{tabular}{|c|c|}
\hline
Question Number & Page Number \\ \hline
1               & 2~-~3         \\ \hline
2               & 4           \\ \hline
3               & 5           \\ \hline
4               & 6~-~8         \\ \hline
5               & 9           \\ \hline
6               & 10          \\ \hline
\end{tabular}%
}
\end{table}
\newpage
\color{red}
\section{Upfront Credit Default Swap}
\color{black}
% 1.a
\bfseries
Traditionally most CDS are traded as a fixed running spread paid through-out the life of the contract. Recently the market has turned towards upfront CDS, where in addition to a (different) fixed running spread there is an immediate (upfront) payment when the deal is entered. This event is referred in the literature as the Big Bang:

”The market for credit-default swaps on corporate and sovereign debt goes through a major overhaul on Wednesday, with changes that will standardize the terms of many of these insurance-like contracts and make them more similar to the bonds they are tied to.

Named the ”Big Bang Protocol” by the International Swaps and Derivatives Association, the raft of reforms will streamline the way the swaps are traded and how they would be settled if bonds or loans default. The adjustments are designed to help facilitate centralized clearing of the swaps, which are used by banks and money managers to hedge their portfolios or to make bets on the performance of companies and countries.”

”For Credit-Default Swaps, Today Comes the Fix-It”, S Ng and E Barrett, Wall Street Journal, April 8, 2009.

In this new formulation, instead of choosing the spread to equate the value of the contract legs to the protection buyer and seller, the spread is fixed at the same level for all contracts and the upfront is chosen as an add-on at the initial time to match again the legs. The recent suggestions in ISDA use just one of two running spreads, 100 bps for investment grade CDS and 500 bps for high yield CDS. The recovery is also restricted similarly to be either 40\%or 20\%. The upfront payment can be negative or positive, based on where the corresponding fair spread would be with respect to the fixed spread and on possible recovery differences.

Task: Develop a modified version of the standard credit default swap pricing formula to take into account the upfront spread.

Hint: Make some field research on the websites of the ISDA (www.isda.org and www.cdsmodel.com). See also: Beumee, Johan G. B., Brigo, Damiano, Schiemert, Daniel and Stoyle, Gareth, Charting a Course Through the CDS Big Bang (April 7, 2009). Available at SSRN: \url{http://ssrn.com/abstract=1374407}.
\mdseries

\vspace{5mm}
\textbf {Ans:}


The pay out per year is $C_i = \$1,000 * 10\% = \$100, i = 1,2,...,10, P_T = \$1,000 $. So the CBB can be decomposed in to 9 ZCBs with maturities from 1 year to 9 years and each has a principal of \$100 and a ZCB with a maturity of 10 years and a principal of \$1,100.

% Question 2
\newpage
\color{red}
\section{Bond pricing equation}
\color{black}

\textbf {We wish to find the approximate value of a cashflow for a floorlet on the one month LIBOR, when using the Vasicek model. Show that this is given by:
$$\max\bigg(r_j - r -\frac{1}{24}( \eta - \gamma r), 0 \bigg)$$
where $r_f$ is the floor rate and r the spot rate. You must start by considering the yield curve power series expression given in the calibration and data analysis lecture. Full working should be given for the series expansion. Do not use an affine solution of the form $\exp(A-rB)$ for this question.}

\vspace{5mm}
\textbf {Ans:}

In the Vasicek Model, $r$ follows
$$\d r = (\eta - \gamma r)dt + \sqrt{\beta}dX$$
The bond price satisfies
$$ \frac{\partial V}{\partial t} + \frac{\beta}{2} \frac{\partial^2 V}{\partial r^2} + (\eta - \gamma r)\frac{\partial V}{\partial r} = rV$$

Let $Y = lnV$,
$$ \frac{\partial Y}{\partial t} = \frac{\partial Y}{\partial V}\frac{\partial V}{\partial t}$$
$$ \frac{\partial Y}{\partial r} = \frac{\partial Y}{\partial V}\frac{\partial V}{\partial r}$$
$$ \frac{\partial ^2 Y}{\partial r^2} = \frac{\partial Y_r}{\partial r} = \frac{1}{V}\cdot \frac{\partial^2 V}{\partial r^2} - (\frac{\partial W}{\partial V} \cdot \frac{\partial V }{\partial r})^2$$
$$ \to Y_t + \frac{\beta}{2} (Y_{rr} + Y_r ^2) + (\eta-\gamma r)Y_r = r$$

Assume $Y_t$ has a taylor expansion form as:
$$ Y(r,t;T) = Y(r,t;T) + a(r)(T-t) + b(r)(T-t)^2 + (T-t)^3 \cdot f_1(t)$$ is true for any $t$, where $f_1(t)$ is a function about $t$, $Y(t,T;T) = 0$. Then we have
$$ 0 = (-a(r) -r) + (-2b(r) + \frac{\beta}{2}a''(r) + (\eta -\gamma r)a'(r)) (T-t) + (T-t)^2 \cdot f_2(t)$$
where $f_2(t)$ is a function about t. 
$$\to a(r) = -r, b(r) = -\frac{1}{2}(\eta - \gamma r)$$

Then we get the power expression for yield curve:
$$ y = -\frac{\ln V}{T-t} = r +\frac{1}{2}(\eta - \gamma r)(T-t) + HOT$$

For one month LIBOR, $T-t = \frac{1}{12}$.

$$ y \approx r+\frac{1}{2}(\eta - \gamma r)(T-t) = r +\frac{1}{24} (\eta - \gamma r)$$

Therefore, the approximate value of a cashflow for a floorlet on one month LIBOR is $\max(r_f - r -\frac{1}{24}(\eta - \gamma r), 0)$.


\newpage
% Question 3
\section{}
\color{red}
\textbf{Consider the spot rate $r$, which evolves according to the popular form
$$dr = u(r) dt + \nu r^\beta dX$$
where $\nu$ and $\beta$ are constants. }

\textbf{Suppose such a model has a steady state transition probability density function $p_\infty(r)$ that satisfies the forward Fokker Planck Equation (aka FKE)}

\textbf{Show that this implies that the drift structure is given by
$$ u(r) = \nu ^2 \beta r^{2\beta-1} + \frac{1}{2} \nu^2 r^{2\beta} \frac{d}{dr}(\ln p_\infty)$$}
\vspace{5mm}
\color{black}
\textbf {Ans:}

Fokker-Planck states that: If
$$dX_t = \mu(X_t,t,)dt + \sigma(X_t,t)dW_t$$
, let $D(x,t) = \sigma^2$, then its density satisfies PDE:
$$\frac{\partial p(x,t)}{\partial t} = \frac{\partial}{\partial x} [\mu(x,t)p(x,t)] + \frac{\partial^2}{\partial x^2}[D(x,t) p(x,t)]$$
Since this model has a steady state transition probability density function$p_\infty(r)$, $\frac{\partial p_\infty(r)}{\partial t} = 0$.

Let $D(r,t) = \frac{\nu^2 r^{2\beta}}{2}$, then we have
$$ \frac{\partial [u(r,t)p_\infty(r)]}{\partial r} = \frac{\partial^2}{\partial r^2} [D(r,t)p_\infty(r)]$$
Taking the integral of both sides,
$$u(r,t) p_\infty(r) =\frac{\partial}{\partial r} [D(r,t)p_\infty(r)] =  D'(r,t) p_\infty(r) + D(r,t)p'_\infty(r)$$
$$ \to u(r,t) = D'(r,t) + D(r,t) \frac{p'_\infty(r)}{p_\infty(r)}$$
$$ \to u(r,t) = \nu ^2 \beta r^{2\beta-1} + \frac{1}{2} \nu^2 r^{2\beta} \frac{d}{dr}(\ln p_\infty(r))$$


% Question 4.1
\newpage
\vspace{5mm}
\section{}
\color{red}
\bfseries
4.a Given the risk neutral expectation of the ZCB pricing with the Fundamental Asset Pricing Formula, express the detailed pseudo code algorithm for the numerical pricing of a Zero Coupon Bond using Monte Carlo Simulation.
\mdseries

\vspace{5mm}
\color{black}
\textbf {Ans:}

The Fundamental Asset Pricing Formula is that the risk neutral expectation of the present conditional on $F_t^-$ hence for the ZCB:
$$ B(t;T) =  A(t) E_t^\theta [ \frac{1}{A(T)}]  = E_t^Q[e^{-\int_t^T r_sds}]$$

$r_t$ is given by:
$$dr(t) = (\mu - \sigma_t\theta_t)dt + \sigma_t (dX_t^P + \theta_tdt), r(0) = r_0$$
$$ r_{t+\Delta t} \approx r_t + (\mu - \sigma_t \theta_t) \Delta t + \sigma \xi_t \sqrt{\Delta t} +\sigma_t \theta_t \Delta t,~~\xi_t \sim N(0,1)$$
$$\to r_{t+\Delta t} \approx r_t + \mu \Delta t + \sigma \xi_t \sqrt{\Delta t},~~\xi_t \sim N(0,1)$$

Pseudo code:

1. Generate N random number $\xi_t$, $\xi_t \sim N(0,1)$.

2. Generate a path of $r_t$ according to the equation above, $\Delta t = (T-t)/N$.

3. Calculate $e^{-\int_t^T r_s ds}$,  $\int_t^T r_s ds$ is calculated by $\sum r_s * (T-t)/N$.

4. Repeat $1 \sim 3$ for M times, M is a large number.

5. $B(t;T)$ is the arithmetic-geometric mean of all the results.

% Question 4.2
\vspace{10mm}
\color{red}
\bfseries
4.b Express the forward Brownian Motion with the RW/BM using market price of risk and bond volatility.
\mdseries

\vspace{5mm}
\color{black}
\textbf {Ans:}
$$ X_t^T = X_t^\theta - \int^t_0 \beta(s;T)ds$$
$X^T$ is the forward brownian motion, is inferred from P-RW to Q-Spot-RN then from Q-Spot-RN to T-fwd-RN.
$$X_t^\theta = X_t^P + \int^t_0 \theta_s ds$$
then $$X_t^T = X_t^P + \int^t_0 \theta_s ds - \int^t_0 \beta(s;T)ds = X_t^P + \int^t_0(\theta_s - \beta(s;T))ds$$


% Question 4.3
\vspace{10mm}
\color{red}
\bfseries
4.c Demonstrate the values of $\d \Lambda_t$ and $\d(1/\Lambda_t)$ (hint: apply Ito)
\mdseries

\vspace{5mm}
\color{black}
\textbf {Ans:}
$$ \Lambda_t = \exp({-\frac{1}{2}\int^t_0(\beta(s;T)^2 ds + \int^t_0 \beta(s;T) dX_s^Q}) $$
$$ Y = \ln X = -\frac{1}{2}\int^t_0(\beta(s;T))^2 ds + \int^t_0 \beta(s;T) dX_s^Q$$
$$\d Y = -\frac{1}{2}(\beta(t;T))^2 + \beta(t;T)dX_s^Q$$
By Ito's lemma,
\begin{equation*}
\begin{aligned}
\d \Lambda_t  &= \d e^{Y_t} = e^{Y_t}\d Y_t+\frac{1}{2} e^{Y_t} (\d Y_t)^2\\
& = \Lambda_t \beta (t,T)dX_t^Q - \frac{1}{2} \Lambda_t \beta^2(t;T)dt + \frac{1}{2} \Lambda_t \beta^2(t,T)dt\\
& = \Lambda_t \beta(t,T)dX_t^Q
\end{aligned}
\end{equation*}

Similarly, we have
\begin{equation*}
\begin{aligned}
\d (1/\Lambda_t)  &= \d e^{-Y_t} = - e^{-Y_t}\d Y_t+\frac{1}{2} e^{-Y_t} (\d Y_t)^2\\
& = - \Lambda_t^{-1} \beta (t,T)dX_t^Q + \frac{1}{2} \Lambda_t^{-1} \beta^2(t;T)dt + \frac{1}{2} \Lambda_t^{-1} \beta^2(t,T)dt\\
& = - \Lambda_t^{-1} \beta(t,T)dX_t^Q + \Lambda_t^{-1} (\beta(t;T))^2 \d t
\end{aligned}
\end{equation*}

\vspace{10mm}
% Question 4.4
\color{red}
\bfseries
4.d Show that an Euro Call on a ZCB is
\begin{equation*}
\left \{
  \begin{aligned}
    &C(t) = B(t;U)N(d_1(B(t;U), t, T)) - K\cdot B(t;T)N_2(d_2(B(t;T), t,T))\\
    &d_1(b,t,T) = \frac{\ln \frac{b}{K} - \ln B(t;T) + \frac{1}{2}\nu^2(t,T)}{\nu(t;T)}  ~~~~d_2(b,t,T) = d_1(b,t,T)-\nu_\theta(t;T) \\
    &\nu^2(t;T) = \int_t^T (\beta(s;U) - \beta(s;T))ˆ2 ds
  \end{aligned} \right.
\end{equation*}
\mdseries

\vspace{5mm}
\color{black}
\textbf {Ans:}
\begin{equation*}
\begin{aligned}
C(t;T) &= A(t)\bigg(E_t^Q\bigg[\frac{I_{B(T;U)>K}}{A(U)}\bigg] - K\cdot E_t^Q\bigg[\frac{I_{B(T,U)<K}}{A(T)}\bigg]\bigg) \\
&= B(t;U) \cdot N(d_1(B(t,U),t,T)) - K\cdot B(t;T), t,T))
\end{aligned}
\end{equation*}

Since the bond volatility $\beta (t;T)$ is a stochastic, we have
$$ \nu_\theta^2 (t;T) = \int^T_t(\beta(s;U) - \beta(s;T))^2 ds$$

Let's look at the original B-S formula,
$$ C = S_t N(d_1) - K \cdot e^{-rt}N(d_2) $$
$$ d_1 = \frac{1}{\sigma \sqrt{T-t}}\bigg[ln\bigg(\frac{S_t}{K}\bigg) + (r + \frac{\sigma^2}{2}(T-t) \bigg]$$
$$ d_2 = d_1 - \sigma \sqrt{T-t} $$

For ZCB,
 $$\sigma \sqrt{T-t} = \sqrt{\nu_\theta^2 (t;T)} = \nu_\theta (t;T) $$
 $$S_t = B(t,U)$$
 $$r(T-t) = \int^T_t r_sds = -\ln B(t;T)$$
 
then we have
$$d_1(b,t,T) = \frac{\ln \frac{b}{K} - \ln B(t;T) + \frac{1}{2}\nu^2(t,T)}{\nu(t;T)}$$
$$d_2(b,t,T) = d_1(b,t,T)-\nu_\theta(t;T)$$

Finally we get
\begin{equation*}
\left \{
  \begin{aligned}
    &C(t) = B(t;U)N(d_1(B(t;U), t, T)) - K\cdot B(t;T)N_2(d_2(B(t;T), t,T))\\
    &d_1(b,t,T) = \frac{\ln \frac{b}{K} - \ln B(t;T) + \frac{1}{2}\nu^2(t,T)}{\nu(t;T)}  ~~~~d_2(b,t,T) = d_1(b,t,T)-\nu_\theta(t;T) \\
    &\nu^2(t;T) = \int_t^T (\beta(s;U) - \beta(s;T))ˆ2 ds
  \end{aligned} \right. 
\end{equation*}

\iffalse
\newpage
% Question 5
\section{}
\color{red}
\bfseries
Consider the Cox, Ingersoll \& Ross model for the spot rate r given by
$$ dr = (\eta - \gamma r) dt + \sqrt{\alpha r} dX$$
with mean rate $\eta/\gamma$ and reversion rate $\gamma$. Suppose $\eta/\gamma = 0.1$ and $\gamma = 0.1$, and diffusion of the process is $\sqrt{\alpha r} = 0.02$. Price a Zero Coupon Bond of the form $Z(r,t,;T) = \exp \{(A(t;T) - rB(t;T)\}$ that matures in year 10, if the spot rate $r= 10\%$: The forms of $A(t;T)$ and $B(t;T)$ provided below

\begin{equation*}
\left.
  \begin{aligned}
    &Z(r,t;T) = e^{A(t;T) - rB(t;T)}\\
    &u(r,t) - \kappa(r,t) \omega(r;t) = \eta(t) - \gamma(t) r\\
    &\omega(r,t) = \sqrt{\alpha(t)r + \beta(t)}
  \end{aligned} \right\}
\end{equation*}

\begin{equation*}
\Rightarrow \left \{
  \begin{aligned}
    &B(t;T)= \frac{2(e^{\psi_1 (t)(T-t)}-1)}{(\gamma(t) +\psi_1(t))(e^{\psi_1(t)(T-t)}-1)+2\psi_1(t)}\\
    &A(t;T) = \frac{2a(t)}{\alpha(t)}\psi_2(t) \ln(a(t)-B(t;T)) - \frac{2a(t)\psi_2(t)}{\alpha(t)}\ln a(t)\\
    & ~~~~~~~~~ +\bigg(\frac{2\psi _2(t)}{\alpha(t)} + \frac{\beta(t)}{\alpha(t)}\bigg) b(t) \ln \bigg(\frac{B(t;T) + b(t)}{b(t)} \bigg) - \frac{B(t;T)\beta(t)}{\alpha (t)}\\
    & \left \{ 
    \begin{aligned}
    &\psi_1(t) = \sqrt{\gamma^2(t) + 2\alpha(t)} ~~~~~ \psi_2(t) = \frac{\eta(t) - \frac{a(t)\beta(t)}{2}}{a(t) + b(t)}\\
    &a(t) = \frac{-\gamma(t) + \psi_1(t)}{\alpha(t)} ~~~~~b(t) = \frac{\gamma(t) + \psi_1(t)}{\alpha(t)}
    \end{aligned} \right.
  \end{aligned} \right. 
\end{equation*}
\mdseries

\vspace{5mm}
\color{black}
\textbf {Ans:}

\begin{figure}[h]
\centering
  \includegraphics[width=0.8\linewidth]{q5.png}
  \caption{$Calculation in Jupyter Notebook$}
  \label{1b}
\end{figure}
Price of the ZCB $= 0.3799$.

\newpage
% Question 6
\color{red}
\section{}
\bfseries
Consider the process given by 
$$ dU_t = -\gamma U_t dt + \sigma dX_t, ~~U_0 =u$$
where $\gamma, \sigma$ are constants and $dX_t$ is an increment in a Wiener process. Solve this equation for $U_t$ and hence write down $E[U_t]$.
\mdseries

\vspace{5mm}
\color{black}
\textbf {Ans:}

\begin{equation*}
\begin{aligned}
dU_t &= -\gamma U_t dt + \sigma dX_t \\ 
d e^{\gamma t}U_t & = e^{\gamma t}dU_t + \gamma e^{\gamma t} U_t dt\\
& = e^{\gamma t}(-\gamma U_t dt + \sigma dX_t) + \gamma e^{\gamma t} U_t dt\\
&= \sigma e^{\gamma t} dX_t\\
e^{\gamma t} U_t - U_0 &= \sigma \int^t_0 e^{\gamma t} dX_s, ~~ U_0 = u\\
U_t & = e^{-\gamma t} u + \sigma \int^t_0 e^{\gamma (s-t)} dX_s
\end{aligned}
\end{equation*}

Then we have
$$ E(U_t) = e^{-\gamma t}u$$
\fi
\end{document}